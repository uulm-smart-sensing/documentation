\documentclass[12pt]{article}

\usepackage[utf8]{inputenc}
\usepackage[T1]{fontenc}
\usepackage[ngerman]{babel}
\usepackage{tabularx}
\usepackage{hyperref}
\usepackage[margin=2cm,a4paper,headsep=1.2cm]{geometry}
\usepackage{ifthen}

\renewcommand{\baselinestretch}{1.1}
\setlength{\parindent}{0pt}
\setlength{\parskip}{1em}

\makeatletter
\newcommand{\labeltext}[2]{%
  \@bsphack
  \csname phantomsection\endcsname % in case hyperref is used
  \def\@currentlabel{#1}{\label{#2}}%
  \@esphack
}
\makeatother    

\newcommand{\prio}[1]{\ifthenelse{\equal{#1}{1}}{low}{\ifthenelse{\equal{#1}{2}}{medium}{\ifthenelse{\equal{#1}{3}}{high}{\textbf{INVALID!}}}}\relax}

\newcounter{fr}
\newcommand{\fr}[8]{
\refstepcounter{fr}\label{#8}
\begin{tabularx}{16cm}{l|X}
 & \textbf{#1} \hfill \textbf{FR\arabic{fr}} \\ \hline
Description & #2\\ \hline
\ifthenelse{\equal{#3}{}}{}{Precondition & #3 \\ \hline}
\ifthenelse{\equal{#4}{}}{}{Postcondition & #4 \\ \hline}
Rationale & #5
\ifthenelse{\equal{#6}{}}{}{\\ \hline Dependencies & #6} 
\ifthenelse{\equal{#7}{}}{}{ \\ \hline Priority & \prio{#7}}
\end{tabularx}
\vspace*{0.75cm}
}

\newcommand{\rref}[1]{\ref{#1}\textsuperscript{$\rightarrow$ p. \pageref{#1}}}
\newcommand{\frref}[1]{FR\ref{#1}\textsuperscript{$\rightarrow$ p. \pageref{#1}}}
\newcommand{\nfrref}[1]{QR\ref{#1}\textsuperscript{$\rightarrow$ p. \pageref{#1}}}

\begin{document}

\section{Functional requirements}
\subsection{Management of devices}
\label{ssec:Management of devices}
\subsubsection{Adding devices}

\fr{get smartphone information}{The library must be able to get information about the smartphone, on which the smart sensing app, which is developed with the library, is installed on.}{The user must allow the app to get the smartphone information. If the user deny, the library must provide a hint, which can be shown up in the app.}{}{The library must collect data about the current smartphone to figure out, which sensors are available and which implementation to use (e. g. based on current OS). But the user must have the control, which data the app respectively the library collect.}{}{3}{FA-get-smartphone-information}

\fr{add smartphone sensors}{The library must provide the opportunity to user to add all sensors available on the smartphone to a list of usable sensors.}{The user must allow the library to discover all sensor by getting the smartphone information and the library got the information about the smartphone.}{All discovered and added sensors can be tracked from now on. The tracking can be started by \glqq{}adding\grqq{} the sensors (c. f. \frref{FA-add-sensor}).}{There must be an option for the user to make the sensors on its smartphone traceable. Otherwise the user would not have the freedom to choose which sensors to use.}{\frref{FA-get-smartphone-information}}{3}{FA-add-smartphone-sensors}

% add similar functional requirements for the wearable devices and the IoT and smart home devices!

\subsubsection{Getting list of devices}

\fr{get supported devices}{The library must provide a list of devices, which are supported by the library. This list must contain all devices, that can be added and their sensors used.}{}{}{There must be an option for the user to select devices to add or to get information, which devices can be used.}{}{2}{FA-get-supported-devices}

\fr{get used devices}{The library knows, which devices are currently used or rather from which devices the currently added sensors come from.}{The device is added to the list of used devices, if the user manually add the device (c. f. \frref{FA-add-smartphone-sensors}).}{After the device is added, it will be contained in the list, the library return, if the used devices are requested by the user.}{There must be an option to check, whether a device is used. This is especially necessary when removing devices.}{\frref{FA-add-smartphone-sensors}}{3}{FA-get-used-devices}

\subsubsection{Removing devices}

\fr{check device is used}{The library can check, whether a device is currently used.}{Therefor the user must provide a device, that will be checked.}{}{To prevent bugs when removing or dealing with already removed or not added devices, it is useful to be able to check, whether this device is used and can be removed or dealt with.}{\frref{FA-get-used-devices}}{3}{FA-check-device-used}

\fr{remove device}{The library must remove a device from the list of used devices.}{The device, which should be removed, must be used.}{The removed device will not longer returned, if the user request the used devices and all sensors, which are available on this device are no longer accessible and especially traceable.}{There must be an option to remove devices respectively all sensors, which are provided by this device, because maybe the user do not want to track the sensors any longer.}{\frref{FA-check-device-used}}{3}{FA-remove-device}

\subsection{Organization of sensors}
\label{ssec:Organization of sensors}
\subsubsection{Adding sensor}

\fr{check sensor name}{The library must check, whether the selected sensor name is valid. A sensor name is valid, if
\begin{itemize}
    \item the name is unique
\end{itemize}
}{}{}{If two or more sensors have the same names, the search for sensors by name is not unambiguous.}{}{3}{FA-check-sensor-name}

\fr{check precision}{The library must check, whether the selected precision is valid. It is valid, if 
\begin{itemize}
    \item precision is an unsigned integer
    \item is between 0 and 10 (0 means, that the data is round to an integer using the standard rounding; 1-10 means the data is round to decimal with 1-10 fractional digits)
\end{itemize}
}{}{}{If the precision is not an unsigned integer, it is not possible to round and if the number is to big, the number gets to long.}{}{2}{FA-check-precision}

\fr{check time interval}{The library must check, whether the selected time interval is valid. It is valid, if 
\begin{itemize}
    \item the number is an unsigned integer
    \item the number is 0 (which means, the user want live data)
    \item the number is between 1 and 59 (the unit for time interval is seconds or minutes)
    \item the number is between 1 and 23 (the unit for time interval is hours)
    \item the number is between 1 and 7 (the unit for time interval is days)
\end{itemize}
}{}{}{If the time interval is set to invalid value, it is not possible to request data from the sensor, which is the main task of this library.}{}{3}{FA-check-time-interval}

\fr{add sensor for tracking}{The library must provide an API for adding a sensor to a list of tracked sensors.}{The sensor, the user want to add, needs to be known to the library. This means this sensor is in a list of implemented sensors.}{After adding the sensor to the list, the library will start tracking the sensor and make the data accessible from this moment on.}{The user must have the opportunity to add sensors, so they will be tracked, because otherwise there is no option to control, which sensors from the devices will be tracked and which not.}{\frref{FA-add-smartphone-sensors}, \frref{FA-check-sensor-name}, \frref{FA-check-precision}, \frref{FA-check-time-interval}}{3}{FA-add-sensor}
% "adding a sensor to a list of tracked sensors" ist wahrscheinlich nicht die beste Formulierung
% "This means this sensor is in a list of implemented sensors." ist wahrscheinlich auch nicht die beste Formulierung

\fr{set sensor name}{In the process of adding a new sensor, the library must provide an opportunity for setting another name for the sensor.}{By default, every sensor will get a name by the library, which is changeable.}{The sensor will be identified by the new name from this moment on.}{The user must have the opportunity to set the name, because different users prefer different names. Furthermore this name is used for searching the sensor, so it is advantageous if the user knows the sensor names.}{\frref{FA-add-sensor}, \frref{FA-check-sensor-name}}{3}{FA-set-name}

\fr{set unit of sensor data}{In the process of adding a new sensor, the library must provide an opportunity for setting the unit of the sensor data.}{ Therefor the library must be able to deal with different units and provide the user a set of units, it can deal with and user can select from.}{From this moment on the data will be displayed in the selected unit.}{The user must have the opportunity to set the units, because different users are used to use different units (e. g. $^\circ C$ vs. $F$ for temperature)}{\frref{FA-add-sensor}}{1}{FA-set-unit}

\fr{set precision of sensor data}{In the process of adding a new sensor, the library must provide an opportunity for setting the precision of the sensor data. Therefor the library must provide an option, the user can set the number of fractional digits.}{}{From this moment on the sensor data will be round to the number set before.}{The user must have the opportunity to set the precision, because different sensors provide data in different precision, but the user maybe want all data in the same precision.}{\frref{FA-add-sensor}, \frref{FA-check-precision}}{2}{FA-set-precision}

\fr{set time interval of sensor data}{In the process of adding a new sensor, the library must provide an opportunity for setting the time interval, in which the data should be \glqq{}requested\grqq{} from the sensor.}{Therefor the library must be able to deal with different time units and provide the user the option to select a time unit and input an unsigned integer. }{This unsigned integer combined with the time unit will represent the time interval and the sensor data will be collected in this time interval.}{The user must have the opportunity to set the frequency of getting the data, because the different sensors do not produce data at the same frequency. Furthermore it is possible, that the user is not interested in getting the sensor data in same regularity for different sensors. For example the user only want to know the temperature once in the morning, but the average activation of the smartphone every hour.}{\frref{FA-add-sensor}, \frref{FA-check-time-interval}}{3}{FA-set-time-interval}

\fr{DRAFT: use microphone as sensor}{The library must provide an API for starting and stopping the use of the internal microphone of the device.}{}{}{Using the microphone can be helpful to determine the surrounding noise.}{}{1}{FA-use-microphone}
\subsubsection{Editing sensor}

\fr{edit properties of sensor}{The library must provide the opportunity to change the following properties of the sensor, which must be set when adding the sensor: 
\begin{itemize}
    \item the unit of the sensor data
    \item the name of the sensor
    \item the precision of the sensor data
    \item the time interval, the sensor data is collected
\end{itemize}
}{The new values set must be valid and the sensor, whose specifications will be changed, must be added before.}{The sensor will be named or provide data matching the changed values.}{The user must have the opportunity to change the properties of the sensor, because users can make mistakes when input values and that's why input must be editable.}{\frref{FA-check-sensor-name}, \frref{FA-check-precision}, \frref{FA-check-time-interval}, \frref{FA-add-sensor}}{2}{FA-edit-sensor}

\subsection{Getting sensor list}

\fr{get sensors}{The library must store a list of all sensors, which were added by the user through the devices and can be started or are currently running. Furthermore this list can be returned on demand.}{}{}{There must be an option to track all sensors and their status, so the user can trace what is happening and search for sensors. Additionally the library must know all sensors for managing them.}{\frref{FA-add-smartphone-sensors}, \frref{FA-add-sensor}}{3}{FA-get-sensors}

\fr{get used sensors}{The library knows, which sensors are currently tracked and can return a list of all sensors, which are currently tracked.}{}{}{There must be an option to get all running sensors for checking the status of a specific sensor.}{\frref{FA-get-sensors}}{1}{FA-get-running-sensors}

\subsubsection{Removing sensor}

\fr{check sensor is used}{The library can check, whether a sensor is currently tracked and started. Therefor the user can provide a sensor name and the sensor, identified by this name, will be checked.}{}{The library return, whether a sensor is currently used (so tracked) or not.}{To prevent bugs when removing or dealing with sensors, it is useful to be able to check, whether this sensor is started and can be stopped or edited.}{\frref{FA-get-running-sensors}}{3}{FA-check-sensor-used}

\fr{stop sensor tracking}{The library must provide an API for removing a sensor from the list of tracked sensors.}{The sensor, the user want to remove, needs to currently run, so started before and known to the library.}{After \glqq{}removing\grqq{} the sensor from the list, the sensor will no longer appear in the list of currently tracked sensors and the tracking of the sensor will be stopped. This means, that the sensor still exist and can be started again (c. f. \frref{FA-add-sensor}), but will not be requested for data anymore.}{The user must have the opportunity to stop tracking sensors, because otherwise there is no option to control, which sensors from the devices will be tracked and which not.}{\frref{FA-add-sensor}, \frref{FA-check-sensor-used}}{3}{FA-remove-sensor}

\subsubsection{Search sensor}

\fr{search by name}{The library must provide an API for searching a sensor by its sensor name. Therefor the API must accept a name (and optional the information, whether only currently tracked or all matching sensors should be returned). If matching sensors exist, they should be returned to the user.}{The provided sensor name is valid and matches any sensor, otherwise there cannot be a list returned.}{}{If the user want to get a specific sensor, searching by name is a simplification for the user because he do not read through all sensors}{\frref{FA-check-sensor-name}}{3}{FA-search-sensor-by-name}

\fr{search by device}{The library must provide an API for searching a sensor by the device it belongs to. Therefor the API must accept a device name (and optional the information, whether only currently tracked or all matching sensors should be returned). If matching devices exist, the sensors (currently active or all depending on the search parameters) should be returned to the user.}{The provided device name is valid and matches any known device, otherwise there cannot be a list returned.}{}{If the user want to get a specific sensor, searching by device is an option, if the user do not know the name, but the device and do not want to search trough all sensors}{\frref{FA-get-supported-devices}}{1}{FA-search-sensor-by-device}

\subsection{Management of data from sensor}
\label{ssec:Management of data from sensor}

\subsubsection{Fetching data from sensor}

\fr{check sensor (data) is available}{The library can check, whether a sensor is available respectively the sensor can provide data.}{The sensor was added and is currently running.}{}{Before data from a sensor is requested, this sensor must be available, because otherwise it cannot provide sensor data.}{\frref{FA-add-sensor}, \frref{FA-add-smartphone-sensors}}{3}{FA-check-sensor-available}

\fr{fetch data from sensor}{The library must provide an API, which is able to deal with a sensor name and fetch the data from the sensor, identified by the provided name.}{The provided sensor name is valid (c. f. \frref{FA-check-sensor-name}) and the sensor, identified by the name, must be available, which must be checked before}{}{The user must be able to get information of the sensors, because they not only collect data but also provide them.}{\frref{FA-check-sensor-available}, \frref{FA-check-sensor-name}}{3}{FA-fetch-sensor-data}

\subsubsection{Saving sensor data}

\fr{save live sensor data}{The library must be able to save (live) data from a sensor into the local storage. Therefor the data will be saved in combination with at least the 
\begin{itemize}
    \item name of the sensor providing this data
    \item the unit the data are represented in
    \item the timestamp, the data were produced
\end{itemize}
}{The sensor, whose data should be saved, must be available and currently running, so it can provide data.}{The data of the sensor will be permanently stored in the local storage and can be read later.}{It is necessary to be able to save data into a storage, because otherwise you cannot export data or get historic data.}{\frref{FA-check-sensor-available}, \frref{FA-fetch-sensor-data}}{3}{FA-save-live-data}

\fr{save high-frequent sensor data}{The library must be able to deal with high-frequent data and save them. In this context, \textit{high-frequent} means, that the sensor provides at least one data point in a second. This type of sensor data needs to be stored in an efficient way, so not every data point individually.}{The sensor, whose data should be saved, must be available and currently running, so it can provide data.}{The data of the sensor will be permanently stored in the local storage and can be unambiguous read later.}{It is necessary to deal with high-frequent data, because they are produced by different sensors like the accelerometer. It is not possible to simply store them, because due to the amount of data points to size of the sensor data would be to large for the local storage or at least not acceptable.}{\frref{FA-check-sensor-available}, \frref{FA-fetch-sensor-data}}{3}{FA-save-high-frequent-data}

\fr{save processed sensor data}{If the user requested processed data, the data in combination with the query will be saved in the storage.}{The query must be valid, so only containing known aggregate functions and retrieving methods from \frref{FA-retrieve-sensor-data-time-interval}, \frref{FA-retrieve-sensor-data-moment-in-time}, \frref{FA-retrieve-sensor-data-by-aggregate-function}}{If a user use the same query again, the data will not be aggregated again, but the stored values will be returned.}{If requests are saved, the data do not need to be re-aggregated, if the user do the same request again.}{\frref{FA-retrieve-sensor-data-time-interval}, \frref{FA-retrieve-sensor-data-moment-in-time}, \frref{FA-retrieve-sensor-data-by-aggregate-function}}{1}{FA-save-processed-data}

\fr{remove sensor data}{The library must provide an API for deleting sensor data points. This means, the user can provide a time interval and a sensor name and all data points from the sensor identified by the name are deleted from the local storage.}{The user need to provide a valid sensor name, whose data the user want to delete. Furthermore the user needs to provide a valid time interval and the sensor needs to have data points in this time interval.}{The requested data will be deleted permanently, so there is no chance to restore the data or to retrieve this data again.}{The user must have the opportunity to delete sensor data because otherwise he would not have the control over the data.}{\frref{FA-check-sensor-name}, \frref{FA-check-time-interval}}{3}{FA-delete-sensor-data}

\subsubsection{Retrieving sensor data}

\fr{retrieve sensor data by specific moment in time}{The library must provide an API, so the user can retrieve the data of a specific sensor at a specific moment in time.}{The user need to provide a sensor, whose data the user want and the sensor needs to have a data point at this moment in time.}{}{The user must have the opportunity to get historic data namely on a specific moment in time.}{\frref{FA-fetch-sensor-data}, \frref{FA-save-live-data}, \frref{FA-save-high-frequent-data}}{3}{FA-retrieve-sensor-data-moment-in-time} 

\fr{retrieve sensor data by time interval}{The library must provide an API, so the user can retrieve the data between to specific moments in time. This means, that a user can provide a begin time and an end time and the library return all collected data between this two times.}{The user need to provide a valid sensor name, whose data the user want. Furthermore the user needs to provide a time interval and the sensor needs to have data points in this time interval.}{}{The user must have the opportunity to get historic data namely a series of historic data in a certain time interval. This can be useful for later evaluation.}{\frref{FA-check-sensor-name}, \frref{FA-check-time-interval}, \frref{FA-fetch-sensor-data}, \frref{FA-save-live-data}, \frref{FA-save-high-frequent-data}}{}{FA-retrieve-sensor-data-time-interval}

\fr{apply aggregate functions on sensor data}{The library must provide an API, so the user can apply aggregate functions on sensor data and retrieve the results. The supported aggregate functions will be 
\begin{itemize}
    \item average / arithmetic mean 
    \item median
    \item maximum 
    \item minimum
    \item sum
    \item count (prerequisite: the user also provide a specific value, which should be counted or nothing, so the amount of data points will be counted)
    \item mode
    \item range, so the distance between the minimum and the maximum
    \item standard deviation
\end{itemize}
All this function only make sense, if they are applied on a series of data points from a time interval. The aggregate functions can be combined, if possible.}{The user need to provide a valid sensor name and a time interval or specific moment in time, which define the data set to user want to process. Furthermore the user needs to provide an aggregate function, which should be also applicable.}{The requested data set will not be manipulated in the storage directly, but there will be a copy which will be manipulated and returned.}{The user must have the opportunity to process the sensor data, which is useful for evaluation.}{\frref{FA-retrieve-sensor-data-moment-in-time}, \frref{FA-retrieve-sensor-data-time-interval}, \frref{FA-check-sensor-name}, \frref{FA-check-time-interval}}{3}{FA-retrieve-sensor-data-by-aggregate-function}

\fr{combine queries}{A query, so a request against the library for sensor data of a specific sensor within a specific time interval or on a specific moment in time (maybe with applied aggregate functions) should be applicable to multiple sensors. This means, if the user provide a set of sensor names, all sensor identified by the names, will get the same provided query. The response of every sensor is bundled into on response.}{The user need to provide valid sensor names. Furthermore the user needs to provide a time interval or specific moment in time and the sensors needs to have data points in this time interval.}{}{When working with the library it can be useful, if you do not need to do the same request for every sensor you are interested in, but use one query on multiple sensors and get a bundle of responses to work with. This is easier for working with the library and can be helpful if you want to compare different sensor data.}{\frref{FA-retrieve-sensor-data-moment-in-time}, \frref{FA-retrieve-sensor-data-time-interval}, \frref{FA-retrieve-sensor-data-by-aggregate-function}}{2}{FA-combine-queries}

\subsubsection{Importing and exporting sensor data} 
\fr{specify import / export format}{The library must provide an opportunity to set the format, in which the sensor data can be imported or will be exported.}{The specified format must be supported by the library.}{If a user want to import or export data, this will be done in the specified format as long as the user do not select another format.}{The specification of this format allows different users to use different formats depending on their preferences.}{}{2}{FA-specify-import-export-format}
 
\fr{import sensor data}{The library must provide an opportunity to import sensor data satisfying a specific format.}{The import format (c. f. \frref{FA-specify-import-export-format}) must be supported by the library.}{The imported data will be added to the local storage (database) and can be requested by the user from this moment on.}{The user maybe want to import data, which were collected earlier and exported before and should be re-imported now.}{\frref{FA-specify-import-export-format}}{2}{FA-import-sensor-data}
 
\fr{export sensor data}{The library must provide an opportunity to export sensor data from the local storage in a specific format. Therefor the user must be able to select a format and a start as well as end time and a sensor name. The library must export all data from the sensor, identified by the name, which are between the start and the end time.}{The import format (c. f. \frref{FA-specify-import-export-format}) must be supported by the library and the input values must be valid.}{The exported data will be on the target volume in the specified format.}{The user maybe want to use the data beyond the app or want archive the data externally.}{\frref{FA-specify-import-export-format}}{2}{FA-export-sensor-data}

\subsection{Demo app}
Additionally the library respectively the smart sensing plugin should be tested. This can be done by implementing a demo app using the library. In general the demo app must be able to demonstrate the successful implementation of all previously listed requirements (c. f. \ref{ssec:Management of devices} - \ref{ssec:Management of data from sensor}). That means, the demo app should ... 
\begin{itemize}
    \item ... be able to get the information about the smartphone it is currently running on and display the information or a hint or error, that getting this information was not possible.
    \item ... select a specific sensor and start tracking the sensor
    \item ... select a specific sensor and edit it properties, so change the unit, sensor name, precision, time interval.
    \item ... select a specific sensor and stop tracking the sensor
    \item ... search for a specific sensor
    \item ... select a specific sensor and display the live data from this sensor
    \item ... select a specific sensor and display processed or historical data from this sensor
\end{itemize}

\end{document}

% https://www.nuclino.com/articles/functional-requirements
%??
% https://savioglobal.com/blog/business-analysis/functional-requirements-document-frd-template-examples/ wirkt auch gut